\documentclass[reprint, amsmath,amssymb, amsthm, aip, cp]{revtex4-2}

% LaTeX Template by Aditya Rao
%=================== HEADER =====================%
\usepackage[left=0.7in, right=0.7in, top=0.7in, bottom=0.7in]{geometry}

\usepackage[T1]{fontenc}
\usepackage[utf8]{inputenc}
\usepackage{setspace}
\usepackage{fancyhdr}

% Other Preamble
\usepackage[leftmargin=1cm, rightmargin=1cm]{mdframed}
\usepackage[colorlinks, allcolors=black]{hyperref}


% Code Formatting
\usepackage{listings}
% Figures & Drawings
\usepackage{graphicx}
\usepackage[export]{adjustbox}
\usepackage{tikz}
\usepackage{float}
\usepackage{caption}

% Physics & Mathematics
\usepackage{physics}
\usepackage{mathtools}
%================================================%

%=================== SETUP ======================%
% Page styling
\pagestyle{fancy}
\pagenumbering{arabic}

% Listings Setup
\lstset{
	basicstyle=\footnotesize,
	numbers=left,
	stepnumber=1,
	showstringspaces=false,
	tabsize=1,
	breaklines=true,
	breakatwhitespace=false
}

% Date Formatting
\def\mydate{\leavevmode\hbox{\the\year-\twodigits\month-\twodigits\day}}
\def\twodigits#1{\ifnum#1<10 0\fi\the#1}
%================================================%

%=================== MACROS =====================%
% The main stuff
\let\origthefigure\thefigure

\newcommand{\thedate}{00.00}
\newcommand{\figbig}{0.9\linewidth}
\newcommand{\figsml}{0.7\linewidth}
\newcommand{\newdate}[1]{\newpage\renewcommand{\thedate}{#1}\section*{Date: \thedate}}
\newcommand{\updatefig}[1]{\renewcommand{\thefigure}{\textbf{\thedate #1.\origthefigure}}\setcounter{figure}{0}}

\newcommand{\updatetab}[1]{\renewcommand{\thefigure}{\textbf{\thedate #1.\origthefigure}}\setcounter{table}{0}}

\newcommand{\updatecountfig}[1]{\setcounter{figure}{#1}}
\newcommand{\updatecounttab}[1]{\setcounter{table}{#1}}

\newcommand{\mdfigure}[2]{
	\begin{mdframed}
	\begin{figure}[H]
		\centering
		\captionsetup{width=\figsml}
		\includegraphics[width=\figsml, max height=0.4\textheight]{#1}
		\caption{#2}
	\end{figure}
	\end{mdframed}
}

\newcommand{\mdtable}[2]{
	\begin{mdframed}
	\begin{figure}[H]
		\centering
		\captionsetup{width=\figsml}
		\includegraphics[width=\figsml, max height=0.4\textheight]{#1}
		\caption{#2}
	\end{figure}
	\end{mdframed}
}

% I like the Griffiths' script r
%Script r (vector)
\newcommand{\brc}{\resizebox{!}{1.3ex}{
    \begin{tikzpicture}[>=round cap]
        \clip (0.085em,-0.1ex) rectangle (0.61em,0.875ex);
        \draw [line width=.2ex, <->, rounded corners=0.13ex] (0.1em,0.1ex) .. controls (0.24em,0.4ex) .. (0.35em,0.8ex) .. controls (0.29em,0.725ex) .. (0.25em,0.6ex) .. controls (0.7em,0.8ex) and (0.08em,-0.4ex) .. (0.55em,0.25ex);
    \end{tikzpicture}
}}

%Script r with a hat (unit vector)
\newcommand{\hrc}{\hat{\brc}}

% References and Sourcing
\newcommand{\figref}[1]{Fig. \ref{#1}}
\newcommand{\source}[1]{\caption*{Source: {#1}} }
%================================================%

%==================== INFO ======================%
\newcommand{\name}{Aditya K. Rao}
\newcommand{\expname}{NMR}
%================================================%

\begin{document}
\nocite{*}
\lhead{\name\vspace{0.1cm}}
\chead{\textbf{Advanced Physics Laboratory} \\ \expname\vspace{0.1cm}}
\rhead{\textbf{LAST UPDATED:} \mydate\vspace{0.1cm}}


\newdate{EXAMPLE}
\mdfigure{example-image}{Example image for templating}

\newdate{02.04}
\updatefig{A}


\mdfigure{../02.04/nmrtest.png}{
	Sketch of r.f. pulse from \cite{manual} as a test if my printing format works as expected. The bottom figure shows r.f. pulses at the LF-HF OUTPUT for RC FILTER set to HF.
}

\newdate{02.07}

\mdfigure{../02.07/setup.jpg}{Initial experimental setup for NMR.}

\newdate{02.11}

\mdfigure{../02.11/nmrtest.png}{Conceptual sketch of r.f. pulses at the PROBE HEAD from the lab manual \cite{manual}.}

\mdfigure{../02.11/nmrtest2.png}{r.f. pulses at the LF-HF Output for RC Filter set to HF from the lab manual \cite{manual}.}

\updatefig{D}

\mdfigure{../02.11/scope_12.png}{Initial Reading from Osciilloscope. Notice Peaks}

\mdfigure{../02.11/scope_13.png}{Scope readings using zoom mode. Clearly we can see the packet.}

\mdfigure{../02.11/scope_14.png}{Another scope reading using zoom mode. Maximum voltage observed $V_{pp} = 31.5 \pm 0.5\,\text{V}$}

\updatefig{E}

\mdfigure{../02.11/fidq.png}{FID Signal (Channel One) from Osciilloscope with Difference Signal From Q Output (Channgel Two)}

\mdfigure{../02.11/fidqzoom.png}{Zoomed in version of the previous figure.}

\mdfigure{../02.11/fidqi.png}{FID Signal (Channel One) from Osciilloscope with Difference Signal From Q Output (Channgel Two) and Difference Signal from I Output (Channel Three)}

\mdfigure{../02.11/fidqizoom.png}{Zoomed in version of the previous figure.}


\updatefig{F}
\mdfigure{../02.11/fidteachspinexpected}{Expected FID and Difference signals from the Teach Spin Cw Laboratory Manual \cite[p.~III-7]{teachspin}}

%\updatetab{A}
\newdate{02.14}

\updatefig{A}
\mdfigure{../02.14/heavyminoil.png}{New Data Obtained from scope for heavy mineral oil. We can clearly see a more distinct pulse width forming as is expected from Fig. 02.11F.1 and from the `spin echo' section of the lab manual \cite{manual}.}


\updatefig{B}
\mdfigure{../02.14/lightminoil.png}{Peak obtained when measuring light mineral oil. Clearly, as compared to Fig. 02.14A.1, there peak is signifianly smaller.}



\newdate{02.25}
\mdfigure{../02.25/HMO1Raw.png}{Raw Pulse data obtained from the Osciilloscope. We will casterate this dataset to fit the tail of the exponential.}


\mdfigure{../02.25/HMO1Tail.png}{Obtained fit using the equation in the graph. Obtained parameters are as follows, $A = 2.0 \pm 70000$ $\tau = -2233 \pm 4$, and $x_0 = 0.0002 \pm 16$ with a $\chi^2_{\text{red}}=0.2$.}

\newdate{02.28}
\updatefig{A}
\mdfigure{../ucat/scope_41.png}{Differerent signal than previous measurements observed. Need to retune and adjust parameters.}

\updatefig{B}
\mdfigure{../ucat/scope_44.png}{Possible Spin Echo observed. First peak is from a $\frac{\pi}{2}$-pulse, second peak is a $\pi$-pulse. The third peak may be the observation of a spin echo.}

\newdate{03.03}
% \updatefig{A}
% \mdfigure{../ucat/scope_45.png}{Scope 45 image, possible spin echo.}
\updatefig{A}
\mdfigure{../ucat/fft.png}{Data obtained from taking the fourier transform of the I signal data. Notice the existance of two distinct peaks. One being the mean and the other being the resonance frequency.}

\mdfigure{../ucat/fft2.png}{Mean removed fourier transform.}

\mdfigure{../ucat/fft3.png}{Mean removed fourier transform (new).}

\updatefig{B}
\mdfigure{../ucat/49dif.png}{Resultant I-signal observed after tuning using fourier transofmr method suggested by SKS.}

\mdfigure{../ucat/49sig.png}{Corresponding signal.}

\newdate{03.04}
\updatefig{A}
\mdfigure{../ucat/scope_50.png}{Result of a $\pi$ pulse with pulse length $\ell_\pi = 6.94\,\mu\text{s}$.}

\updatefig{B}
\mdfigure{../ucat/scope_54.png}{Relavent output with new pulse length parameters.}

\updatefig{D}
\mdfigure{../ucat/scope_52.png}{Output signal from distilled water sample. Notice that the signal strength is quite low and very noisy.}

\newpage
\updatefig{F}
\mdfigure{../ucat/opamp.jpg}{Breadboard of non-inverting op-amp.}

\newpage

\updatefig{G}
\mdfigure{../ucat/scope_54.png}{Noisy small signal observed.}

\mdfigure{../ucat/56sig.png}{Observation for Distilled Water.}

\mdfigure{../ucat/57sig.png}{Observation for Distilled Water.}

\mdfigure{../ucat/58sig.png}{Observation for Distilled Water.}

\updatefig{H}
\mdfigure{../ucat/60sig.png}{Observation for Light Mineral Oil.}

\mdfigure{../ucat/61sig.png}{Observation for Light Mineral Oil.}

\mdfigure{../ucat/62sig.png}{Observation for Light Mineral Oil.}

\updatefig{I}
\mdfigure{../ucat/64sig.png}{Observation for Heavy Mineral Oil.}

\mdfigure{../ucat/65sig.png}{Observation for Heavy Mineral Oil.}

\mdfigure{../ucat/67sig.png}{Observation for Heavy Mineral Oil.}

\updatefig{J}
\mdfigure{../ucat/68sig.png}{Playing around with pulses to see if we can implement a unitary gate. This is more just exploration with no expectation of results.}



\newdate{03.05}
\updatefig{A}
\mdfigure{../../Analysis/figures/lmofid.png}{$A = 125.02730736861422 \pm inf$, $\tau = -159.21440895422927 \pm inf$, $x_0 = -0.711144259571313 \pm inf$ $\chi^2_{\text{red}}$ = 2.0}



\updatefig{B}
\mdfigure{../../Analysis/figures/hmofid.png}{$A = 0.0 \pm 9000.0$, $\tau = -5456.4 \pm 90.0$, $x_0 = 0.0 \pm 200.0$, $\chi^2_{\text{red}}$ = 0.003}


\updatefig{C}
\mdfigure{../../Analysis/figures/h2ofid.png}{$A = 0.650974392642326 \pm 1436422.7844434683$, $\tau = 2345.446650108936 \pm 224.61004122688507$, $x_0 = 0.0003897026532702442 \pm 937.2177888340759$, $\chi^2_{\text{red}}$ = 2.0}

\clearpage
\appendix
\bibliography{references.bib}
\end{document}